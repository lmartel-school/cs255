\documentclass{article}[12pt]
\usepackage{color}
\usepackage[normalem]{ulem}
\usepackage{times}
\usepackage{fullpage}
\usepackage{amsmath}
\usepackage{amssymb}
\def \R {\mathbb R}
\def \imp {\Longrightarrow}
\def \eps {\varepsilon}
\def \Inf {{\sf Inf}}
\newenvironment{proof}{{\bf Proof.  }}{\hfill$\Box$}
\newtheorem{theorem}{Theorem}[section]
\newtheorem{definition}{Definition}[section]
\newtheorem{corollary}{Corollary}[section]
\newtheorem{lemma}{Lemma}[section]
\newtheorem{claim}{Claim}[section]
\setlength {\parskip}{2pt}
\setlength{\parindent}{0pt}

\newcommand{\headings}[4]{\noindent {\bf CS154: Introduction to Automata and
Complexity Theory} \hfill {Winter 2015} \\
{\noindent {\bf Homework} #1} \hfill {{\bf Due Date:} #2} \\

\rule[0.1in]{\textwidth}{0.025in}
}

\newcommand{\klnote}[1]{{\color{red} #1}}
\newcommand{\klsout}[1]{{\color{red} \sout{#1}}}

% My custom stuff:
\usepackage{tikz,enumerate}
\newcommand{\soln}{\textbf{\newline Solution: }}
\renewcommand{\emptyset}{\varnothing}
\newcommand{\xor}{\bigoplus}

\title{Programming Project 1 - Short Answers}
\author{Leo Martel, Christine Li}
\date{Friday, Feb 13, 2015}

\begin{document}

\maketitle

\begin{enumerate}[1.]
\item The adversary is prevented from learning information about the
  lengths of the passwords because the passwords are all padded to 64
  bytes (maximum password length), salted with HMAC(domain), which is
  always 256 bits, and finally salted with a random additional salt of
  constant length (64 bits). AES is a block cipher, so given a
  fixed-size input will produce a fixed-size output. Therefore the
  encrypted passwords are all the same length, and no length
  information is leaked.

\item We use the HMAC'd domain (the keyring key) as part of the
  password (keyring value) salt. When $get(domain)$ is called, the
  salted password is decrypted and the stored domain is compared to
  the parameter. If the domain does not match, we throw a tampering
  exception. Since a key-value store by definition only contains one
  password for each domain, swap attacks are impossible.

\item Yes, the trusted location is necessary. If we store the digest
  on disk, an adversary could modify the database and then modify the
  digest to match, causing our integrity check to produce a false
  positive. For example, the adversary could successfully execute a
  rollback attack by rolling back both the database and the digest to
  corresponding earlier versions.

\item Using another secure MAC would not jeopardize the security of
  this system. Specifically, using an arbitrary secure MAC would not
  reveal any information about the domains. As proved in class, secure
  MACs are immune to chosen plaintext attacks. If an attacker could
  mine information about the domains then they could use it to win the
  secure MAC game by existential forgery.

\item We could pad the password manager with bogus entries such that
  it is always at its maximum capacity (or at some practical “large
  enough” capacity). Before encryption, we could put in an indicator
  in either the URL or password that indicates whether the entry is
  real or fake. Thus, when adding a new entry, we could iterate
  through all of the keys until we find a fake one, then delete
  it. Then, we could add in our new entry to bring the total record
  count back up to the maximum.

\end{enumerate}

\end{document}
