\documentclass{article}[12pt]
\usepackage{color}
\usepackage[normalem]{ulem}
\usepackage{times}
\usepackage{fullpage}
\usepackage{amsmath}
\usepackage{amssymb}
\def \R {\mathbb R}
\def \imp {\Longrightarrow}
\def \eps {\varepsilon}
\def \Inf {{\sf Inf}}
\newenvironment{proof}{{\bf Proof.  }}{\hfill$\Box$}
\newtheorem{theorem}{Theorem}[section]
\newtheorem{definition}{Definition}[section]
\newtheorem{corollary}{Corollary}[section]
\newtheorem{lemma}{Lemma}[section]
\newtheorem{claim}{Claim}[section]
\setlength {\parskip}{2pt}
\setlength{\parindent}{0pt}

\newcommand{\headings}[4]{\noindent {\bf CS154: Introduction to Automata and
Complexity Theory} \hfill {Winter 2015} \\
{\noindent {\bf Homework} #1} \hfill {{\bf Due Date:} #2} \\

\rule[0.1in]{\textwidth}{0.025in}
}

\newcommand{\klnote}[1]{{\color{red} #1}}
\newcommand{\klsout}[1]{{\color{red} \sout{#1}}}

% My custom stuff:
\usepackage{tikz,enumerate}
\newcommand{\soln}{\textbf{\newline Solution: }}
\renewcommand{\emptyset}{\varnothing}
\newcommand{\xor}{\bigoplus}

\title{Programming Project 1 - Short Answers}
\author{Leo Martel}
\date{Wednesday, Mar 11, 2015}

\begin{document}

\maketitle

\begin{enumerate}[1.]

\item This scheme is clearly insecure, because the client's response to the server's challenge does not require knowledge of the client's secret key. $SHA-256$ is a deterministic, publicly-available algorithm; an adversary could simply connect to the server and respond with $SHA-256(l)$ to impersonate any client it wishes, violating mutual authentication.

\item By pinning $cs255ca.pem$, our client trusts the cs255 CA (and no other CA). Since the TLS client only accepts connections to servers with certificates signed by this CA, any server without such a cert will be rejected at the $tls.connect$ step. Thus the only possible issuer in the ``field validation'' step is $cs255ca$, which we trust has no reason to lie in its certificates, so we can safely assume the ``issuer'' field is correct without checking it.

\item We could put the ``salt'' half of the private key directly into setup_cipher, rather than running it through the KDF with the password first. This would still run, since the salt is a length $128$ bitarray just like the sk_der is, but this implementation would ignore the client's password. The scheme would be vulnerable to an attacker compromising the client's computer and stealing their private key; the correct implementation requires that an attack also knows or guesses the client's password (which should not be stored anywhere on disk).

% TODO ANOTHER THING


% TODO. AES key reuse?()

\item

\begin{enumerate}[(a)]
\item One simple advantage of symmetric key encryption here is that it is faster. Since we need one encryption and one decryption for each connection, using either RSA (in which encryption is slow) or secure ElGamal (in which neither step is that fast) will be slower than using a simple symmetric-key MAC. This could be a big advantage, especially if our protocol needs to support many short-lived connections.


% One advantage of a (secure) symmetric key protocol is that compromise of a secret key belonging to client $C$ only allows an attacker to impersonate $C$ when connecting to \textbf{one} server, rather than all servers. As discussed below, the current client uses one private key but would be forced to use a distinct key for every server under a symmetric-key model.

\item The disadvantage of symmetric-key challenge-response is that in order to ensure security, a client must use a distinct key with each server it wishes to connect to. It's easy to see why: if a client $C$ shares a symmetric key $k$ with two servers $A$ and $B$, then server $B$ can connect to $A$ by impersonating $C$ using $k$, because by definition $A$ and $C$ both have access to $k$ undera symmetric-key protocol. Since $A \ne C$ can impersonate $C$ when connecting to $B$, the mutual authentication of the server is violated.

Thus, each client must store a distinct secret key for each server it wishes to connect to. On the other hand, under the public-key based system each clients needs only to store a single private key to connect to any number of servers securely. Symmetric key grows this constant space requirement to a linear one, a disadvantage.
\end{enumerate}

\item

\begin{enumerate}[(a)]
\item The server would need to observe standard password best-practices; rather than storing the password in plaintext, the server would store a unique salt and a hashed, salted password for each user. This way, if the server's database is compromised, an attacker cannot impersonate all of the server's clients. They would first need to brute force the hashes--and due to the salting, each brute force attack would only reveal one client's password, even if some clients chose the same password.

\item Using the bad certificate, the attacker connects to the client, claiming to be the server. The client checks the bad certificate, which validates, and then sends their password to the attacker. The attacker now simply records the plaintext of the password.

Later on, the attacker can connect to the server directly and impersonate the client by using the stolen password. The bad cert is no longer needed!
\end{enumerate}

\end{enumerate}

\end{document}
